\documentclass[10pt]{article}

% Lines beginning with the percent sign are comments
% This file has been commented to help you understand more about LaTeX

% DO NOT EDIT THE LINES BETWEEN THE TWO LONG HORIZONTAL LINES

%---------------------------------------------------------------------------------------------------------

% Packages add extra functionality.
\usepackage{times,graphicx,epstopdf,fancyhdr,amsfonts,amsthm,amsmath,algorithm,algorithmic,xspace,hyperref}
\usepackage[left=1in,top=1in,right=1in,bottom=1in]{geometry}
\usepackage{sect sty}   %For centering section headings
\usepackage{enumerate}  %Allows more labeling options for enumerate environments 
\usepackage{epsfig}
\usepackage[space]{grffile}
\usepackage{booktabs}
\usepackage{forest}
\usepackage{enumitem}   
\usepackage{fancyvrb}
\usepackage{todonotes}

% This will set LaTeX to look for figures in the same directory as the .tex file
\graphicspath{.} % The dot means current directory.

\pagestyle{fancy}

\lhead{Final Project}
\rhead{\today}
\lfoot{CSCI 334: Principles of Programming Languages}
\cfoot{\thepage}
\rfoot{Spring 2024}

% Some commands for changing header and footer format
\renewcommand{\headrulewidth}{0.4pt}
\renewcommand{\headwidth}{\textwidth}
\renewcommand{\footrulewidth}{0.4pt}

% These let you use common environments
\newtheorem{claim}{Claim}
\newtheorem{definition}{Definition}
\newtheorem{theorem}{Theorem}
\newtheorem{lemma}{Lemma}
\newtheorem{observation}{Observation}
\newtheorem{question}{Question}

\setlength{\parindent}{0cm}

%---------------------------------------------------------------------------------------------------------

% DON'T CHANGE ANYTHING ABOVE HERE

% Edit below as instructed

\title{Basic BASIC Language Specification} % Replace SnappyLanguageName with your project's name

\author{Maddy Wu \and Amir Estejab} % Replace these with real partner names.

\begin{document}
  
\maketitle

\subsection*{Introduction}

As student's that took intro to CS at Williams, we learned how to program with Python. There were, of course, many benefits to learning 
how to code this way. For example, Python syntax, at the level that we were learning it, was not difficult to learn. However, there were 
also many challenges -- especially for individuals that have never coded before or have only had experience coding with languages like 
Scratch. One of the main challenges that we came across as students in 134, was trying to gain a solid foundational understanding of the 
specifications in a language. Python is obviously a large language with many specifications which is daunting to people that have never
programmed before. \\
\\
BASIC is a programming language that was originally designed to remedy this problem. Designed by John G. Kemeny and Thomas E. Kurtz at 
Dartmouth College in 1963 as a way to make it easy for non-STEM students to learn how to code. For our project, we are looking to create 
an even simpler version of the language that will hopefully further lower the barriers of entry to learning how to the program and really 
reinforce what we believe to be the foundations of programming. 

\subsection*{Design Principles}

Because Basic BASIC is meant to be accessible even to most non-technical of users, both our primitive types and our combining forms will 
try to resemble plain English as much as possible. Furthermore, Basic BASIC will have even fewer specifications than BASIC. Having a 
small set of primitive types and limited data structures, allows the user to gain a stronger understanding of "foundational" concepts which 
will lend itself nicely if the user chooses to continue learning how to program. 

\subsection*{Examples}

Note: The file extensions are a work in progress

\begin{enumerate}
    \item
    test1.txt:

    \begin{verbatim}
    PRINT "Hello World"
    \end{verbatim}

    Run ”dotnet run test1.txt” in the project file. The output should be the following:

    \begin{verbatim}
    "Hello World"
    \end{verbatim} 


    \item
    test2.txt:

    \begin{verbatim}
    PRINT "gibberish"
    \end{verbatim}

    Run ”dotnet run test2.txt” in the project file. The output should be the following:

    \begin{verbatim}
    "gibberish"
    \end{verbatim}

    \item
    example-3.bb: 

    \begin{verbatim}
    "Programming is cool"
    \end{verbatim}

    Run ”dotnet run test3.txt” in the project file. The output should be the following:

    \begin{verbatim}
    "Programming is cool"
    \end{verbatim}

    \item
    test4.txt:

    \begin{verbatim}
    10 REM TI-STEM
    20 CALL CLEAR
    30 RANDOMIZE
    40 B=0
    50 R$=" "
    60 PRINT "Do you want to play with"
    70 PRINT "1:letters?"
    80 PRINT "2:numbers?"
    90 PRINT "3:0 an 1 only?"
    100 PRINT "4:or do you want to stop?"
    110 CALL KEY (5,K,S)
    120 IF S = 0 THEN 110
    130 IF K > 57 THEN 110 
    140 IF K < 48 THEN 110
    150 K=K-48 
    160 ON K GOSUB 420,450,480,510
    170 INPUT "Difficulty? (1-6)":DIF
    180 REM  DISPLAY************
    190 B=B+1
    200 R$=" "
    210 CALL CLEAR 
    220 FOR CC1=1 TO B
    230 A=INT(RND*NR)+W
    240 CALL HCHAR(12,16,A)
    250 R$=R$&CHR$(A)
    260 FOR CC2=1 TO 100/DIF
    270 NEXT CC2
    280 CALL HCHAR(12,16,31)
    290 FOR CC2=1 TO 100/DIF
    300 NEXT CC2
    310 NEXT CC1
    320 REM INPUT ANSWER *******
    330 INPUT "ANSWER: ":AN$
    340 IF AN$=R$ THEN 190
    350 PRINT "WRONG"
    360 PRINT "YOUR SCORE IS:";B-1
    370 PRINT "THE ANSWER IS:";R$
    380 PRINT "PRESS ANY KEY"
    390 CALL KEY (5,K,S)
    400 IF S=0 THEN 390
    410 GOTO 20
    420 NR=26
    430 W=65
    440 RETURN
    450 NR=10
    460 W=48
    470 RETURN 
    480 NR=2
    490 W=48
    500 RETURN
    510 PRINT "End of this game"
    520 END
    \end{verbatim}

    Run ”dotnet run test1.txt” in the project file. The output should be the following:

    \begin{verbatim}
    "Hello World"
    \end{verbatim} 


    \item
\end{enumerate}

\subsection*{Language Concepts}

For now, the user needs only to understand strings and print statements. Strings are a primitive data type in this language meaning they 
cannot be broken down into constiuent parts. The print statment, on the other hand, is a combining form that acts as an 'operation' which 
takes a string and does something with it (in this case, prints it on screen). \\
\\
As we continue with our language, we will incorporate other elements of the language like GO TO or INPUT. Once again, the goal of Basic 
BASIC is to be easy to learn and program with. All types be it primitives or combining forms should resemble plain English as much as 
possible. 

\subsection*{Formal Syntax}

\begin{verbatim}
<Expr>     := <Command>_<String>
           | <String>
           | \epsilon
<Command>  := PRINT
<String>   := ""
\end{verbatim}

\subsection*{Semantics}
The 1.0 specification of Basic BASIC is very limited in its scope. It gives users one primitive type (string) and one combining form (PRINT)
that allows for printing of different strings. With the current minimally working version, all valid expressions under the BNF prints out
the AST. The language can read in one line from a file (i.e.: "PRINT "hello world"") as input and will output the AST with the 
prettyprint function as a result on the terminal.\\
\\
As we continue to build the language we will likely also incorporate numbers or integers and 
characters. The combining forms would be PRINT, INPUT, IF ELSE, and maybe GO TO. We will also want a way to read in multiple lines from a 
file because a program written in Basic BASIC should have multiple lines. Given that we eventually do want to implement an INPUT function, 
our language will have the ability to read input. \\
\\
The ultimate goal of Basic BASIC is to be able to operate the way Python might in 134. We want to be able to create very elementary programs
that do whatever the user may want them to do. One example migh tbe if a user wants to create a mini video game using Basic BASIC. That would 
certainly be a possibility. Another possibilty, though we might not be able to implement it quite yet, is the ability to create graphics using
Basic BASIC while still using the same few primitives and combining forms that Basic BASIC has.

% DO NOT DELETE ANYTHING BELOW THIS LINE
\end{document}