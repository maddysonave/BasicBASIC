\documentclass[10pt]{article}

% Lines beginning with the percent sign are comments
% This file has been commented to help you understand more about LaTeX

% DO NOT EDIT THE LINES BETWEEN THE TWO LONG HORIZONTAL LINES

%---------------------------------------------------------------------------------------------------------

% Packages add extra functionality.
\usepackage{times,graphicx,epstopdf,fancyhdr,amsfonts,amsthm,amsmath,algorithm,algorithmic,xspace,hyperref}
\usepackage[left=1in,top=1in,right=1in,bottom=1in]{geometry}
\usepackage{sect sty}   %For centering section headings
\usepackage{enumerate}  %Allows more labeling options for enumerate environments 
\usepackage{epsfig}
\usepackage[space]{grffile}
\usepackage{booktabs}
\usepackage{forest}
\usepackage{enumitem}   
\usepackage{fancyvrb}
\usepackage{todonotes}

% This will set LaTeX to look for figures in the same directory as the .tex file
\graphicspath{.} % The dot means current directory.

\pagestyle{fancy}

\lhead{Final Project}
\rhead{\today}
\lfoot{CSCI 334: Principles of Programming Languages}
\cfoot{\thepage}
\rfoot{Spring 2024}

% Some commands for changing header and footer format
\renewcommand{\headrulewidth}{0.4pt}
\renewcommand{\headwidth}{\textwidth}
\renewcommand{\footrulewidth}{0.4pt}

% These let you use common environments
\newtheorem{claim}{Claim}
\newtheorem{definition}{Definition}
\newtheorem{theorem}{Theorem}
\newtheorem{lemma}{Lemma}
\newtheorem{observation}{Observation}
\newtheorem{question}{Question}

\setlength{\parindent}{0cm}

%---------------------------------------------------------------------------------------------------------

% DON'T CHANGE ANYTHING ABOVE HERE

% Edit below as instructed

\title{Basic BASIC Language Specification} % Replace SnappyLanguageName with your project's name

\author{Maddy Wu \and Amir Estejab} % Replace these with real partner names.

\begin{document}
  
\maketitle

\subsection*{Introduction}

As student's that took intro to CS at Williams, we learned how to program with Python. There were, of course, many benefits to learning 
how to code this way. For example, Python syntax, at the level that we were learning it, was not difficult to learn. However, there were 
also many challenges -- especially for individuals that have never coded before or have only had experience coding with languages like 
Scratch. One of the main challenges that we came across as students in 134, was trying to gain a solid foundational understanding of the 
specifications in a language. Python is obviously a large language with many specifications which is daunting to people that have never
programmed before. \\
\\
BASIC is a programming language that was originally designed to remedy this problem. Designed by John G. Kemeny and Thomas E. Kurtz at 
Dartmouth College in 1963 as a way to make it easy for non-STEM students to learn how to code. For our project, we are looking to create 
an even simpler version of the language that will hopefully further lower the barriers of entry to learning how to the program and really 
reinforce what we believe to be the foundations of programming. 

\subsection*{Design Principles}

Because Basic BASIC is meant to be accessible even to most non-technical of users, both our primitive types and our combining forms will 
try to resemble plain English as much as possible. Furthermore, Basic BASIC will have even fewer specifications than BASIC. Having a 
small set of primitive types and limited data structures, allows the user to gain a stronger understanding of "foundational" concepts which 
will lend itself nicely if the user chooses to continue learning how to program. 

\subsection*{Examples}
\begin{enumerate}
   
    \item
    test1.bbas:

    \begin{verbatim}
    10 PRINT 2^3 
    \end{verbatim}

    Run ”dotnet run test1.bbas” in the project file. The output should be the following:

    \begin{verbatim}
    8
    \end{verbatim}

    \item
    test-2.bbas: 

    \begin{verbatim}
    10 INPUT "Enter the first number: ", num1
    20 INPUT "Enter the second number: ", num2
    30 PRINT "The sum is: "; num1 + num2
    \end{verbatim}

    Run ”dotnet run test2.bbas” in the project file. The user will then be prompted twice for a number. 
    Once the user inputs both numbers, say the user inputs 10 and 20, the output should be:\\

    \begin{verbatim}
    "The sum is: 30"
    \end{verbatim}

    \item
    STM.bbas:

    \begin{verbatim}
    10  REM TI-STEM
    20  CALL CLEAR
    30  RANDOMIZE
    40  B=0
    50  R$=" "
    60  PRINT "Do you want to play with"
    70  PRINT "1:letters?"
    80  PRINT "2:numbers?"
    90  PRINT "3:0 an 1 only?"
    100 PRINT "4:or do you want to stop?"
    110 CALL KEY (5,K,S)
    120 IF S = 0 THEN 110
    130 IF K > 57 THEN 110 
    140 IF K < 48 THEN 110
    150 K=K-48 
    160 ON K GOSUB 420,450,480,510
    170 INPUT "Difficulty? (1-6)":DIF
    180 REM  DISPLAY************
    190 B=B+1
    200 R$=" "
    210 CALL CLEAR 
    220 FOR CC1=1 TO B
    230 A=INT(RND*NR)+W
    240 CALL HCHAR(12,16,A)
    250 R$=R$&CHR$(A)
    260 FOR CC2=1 TO 100/DIF
    270 NEXT CC2
    280 CALL HCHAR(12,16,31)
    290 FOR CC2=1 TO 100/DIF
    300 NEXT CC2
    310 NEXT CC1
    320 REM INPUT ANSWER *******
    330 INPUT "ANSWER: ":AN$
    340 IF AN$=R$ THEN 190
    350 PRINT "WRONG"
    360 PRINT "YOUR SCORE IS:";B-1
    370 PRINT "THE ANSWER IS:";R$
    380 PRINT "PRESS ANY KEY"
    390 CALL KEY (5,K,S)
    400 IF S=0 THEN 390
    410 GOTO 20
    420 NR=26
    430 W=65
    440 RETURN
    450 NR=10
    460 W=48
    470 RETURN 
    480 NR=2
    490 W=48
    500 RETURN
    510 PRINT "End of this game"
    520 END
    \end{verbatim}

    Run ”dotnet run STM.bbas” in the project file. The output should be the following:

    \begin{verbatim}
    DO YOU WANT TO PLAY WITH
    1. LETTERS?
    2. NUMBERS?
    3. 0 OR 1 ONLY?
    4. OR DO YOU WANT TO STOP?
    \end{verbatim} 

    How STM works: \\

    Now you make your choice by entering 1, 2, 3, or 4 and also choose a level of difficulty. Say you entered 1 to play with letters. 
    A letter will now appear on the screen but only for a very short time. You have to enter that letter. The computer will then show
    you two letters which you have to enter, then three, and so on. Obviously, as the number of letters increases, remembering them 
    all becomes more difficult. What is the longest string of characters you can remember? 
\end{enumerate}

\subsection*{Language Concepts}

In order to use BASIC Basic the user should have an understanding of basic $($haha$)$ math operations.They also need to have some an 
understanding of strings and print statements. Strings are a primitive data type in this language meaning they cannot be broken down 
into constiuent parts. The print statment as well as the arithmetic operators, on the other hand, are combining forms. \\
\\
As we continue with our language, we will incorporate other elements of the language like GO TO or INPUT. Once again, the goal of Basic 
BASIC is to be easy to learn and program with. All types be it primitives or combining forms should resemble plain English as much as 
possible. 

\subsection*{Formal Syntax}

\begin{verbatim}
<Expr>     := <Command>_<String> | <String> | <Num> |
              (<Expr> + <Expr>) | (<Expr> - <Expr>) | 
              (<Expr> * <Expr>) | (<Expr> / <Expr>) | 
              (<Expr> ^ <Expr>) | \epsilon | <Paren>
<Command>  := PRINT
<String>   := ""
<Num>      := n\in\Z
\end{verbatim}

\subsection*{Semantics}

    \begin{tabular}{ |p{3cm}|p{3cm}|p{3cm}|p{3cm}|p{3cm}|  }
    \hline
    \multicolumn{5}{|c|}{Semantics} \\
    \hline
    Syntax& Abstract Syntax& Type& Prec./Assoc.& Meaning \\
    \hline
    "Hello World" &Bstring of string &string &n/a & A sequence of characters enclosed in double quotes ("). It is a primitive \\
    \hline
    n &Num of int &int &n/a &N is any positive or negative integer. It is a primitive. \\
    \hline
    + &Plus of Expr * Expr &char &left-associative & Adds the value of the left expression to the value of the right expression.\\
    \hline
    \end{tabular}


\begin{enumerate}
    \item
    What are the primitive kinds of values in your system? For example, a primitive might be a number, 
    a string, a shape, or a sound. Every primitive should be an idea that a user can explicitly state 
    in a program written in your language. 
    \begin{enumerate}
        \item
        The primitives that we currently have are strings and numbers.
    \end{enumerate} 
    \item 
    What are the combining forms in your language? In other words, how are values combined in
    a program? For example, your system might combine primitive “numbers” using an operation
    like “plus.” Or perhaps a user can arrange primitive “notes” within a “sequence.”
    \begin{enumerate}
        \item 
        The combining forms we have are your typical arithmetic operators like addition, subtratction,
         * multiplication, division, and exponentiation.
    \end{enumerate} 
    \item 
    How is your program evaluated? In particular
    \begin{enumerate}
        \item 
        Do programs in your language read any input?
        \begin{enumerate}
            \item
            Our language does take in user input. Our language is a way to simplify programming, 
            so users will be programming using our language.
        \end{enumerate}
        \item
        What is the effect (output) of evaluating a program? Does the language produce a file or
        print something to the screen? Use one of your example programs to illustrate what you
        expect as output.
        \begin{enumerate}
            \item
            Depending on the program that the user writes, Basic BASIC should be able to interpret any
            program written in the language and output what the user is hoping to output. The most 
            complex example we have provided is a game called STM which should generate a playable game. 
        \end{enumerate}
    \end{enumerate}
\end{enumerate}


% DO NOT DELETE ANYTHING BELOW THIS LINE
\end{document}