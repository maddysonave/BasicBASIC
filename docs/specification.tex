\documentclass[10pt]{article}

% Lines beginning with the percent sign are comments
% This file has been commented to help you understand more about LaTeX

% DO NOT EDIT THE LINES BETWEEN THE TWO LONG HORIZONTAL LINES

%---------------------------------------------------------------------------------------------------------

% Packages add extra functionality.
\usepackage{times,graphicx,epstopdf,fancyhdr,amsfonts,amsthm,amsmath,algorithm,algorithmic,xspace,hyperref}
\usepackage[left=1in,top=1in,right=1in,bottom=1in]{geometry}
\usepackage{sect sty}	%For centering section headings
\usepackage{enumerate}	%Allows more labeling options for enumerate environments 
\usepackage{epsfig}
\usepackage[space]{grffile}
\usepackage{booktabs}
\usepackage{forest}
\usepackage{enumitem}   
\usepackage{fancyvrb}
\usepackage{todonotes}

% This will set LaTeX to look for figures in the same directory as the .tex file
\graphicspath{.} % The dot means current directory.

\pagestyle{fancy}

\lhead{Final Project}
\rhead{\today}
\lfoot{CSCI 334: Principles of Programming Languages}
\cfoot{\thepage}
\rfoot{Spring 2024}

% Some commands for changing header and footer format
\renewcommand{\headrulewidth}{0.4pt}
\renewcommand{\headwidth}{\textwidth}
\renewcommand{\footrulewidth}{0.4pt}

% These let you use common environments
\newtheorem{claim}{Claim}
\newtheorem{definition}{Definition}
\newtheorem{theorem}{Theorem}
\newtheorem{lemma}{Lemma}
\newtheorem{observation}{Observation}
\newtheorem{question}{Question}

\setlength{\parindent}{0cm}

%---------------------------------------------------------------------------------------------------------

% DON'T CHANGE ANYTHING ABOVE HERE

% Edit below as instructed

\title{Basic BASIC Language Specification} % Replace SnappyLanguageName with your project's name

\author{Maddy Wu \and Amir Estejab} % Replace these with real partner names.

\begin{document}
  
\maketitle

\subsection*{Introduction}

BASIC is a programming language that was originally designed to remedy this problem. Designed by John G. Kemeny and Thomas E. Kurtz at 
Dartmouth College in 1963 as a way to make it easy for non-STEM students to learn how to code. For our project, we aim to create an even 
simpler version of the language that will hopefully further lower the barriers of entry to learning how to the program and really reinforce
what we believe to be the foundations of programming. 

As student's that took intro to CS at Williams, we learned how to program with Python. There were, of course, many benefits to learning 
how to code this way. However, there were also many challenges -- especially for individuals that have never coded before or have only 
had experience coding with languages like Scratch. One of the main challenges that we came across as students in 134, was trying to gain 
a solid foundational understanding of the specifications in a language. Python is obviously a large language with many specifications which 
is daunting to people that have never programmed before. 




\subsection*{Design Principles}

Basic BASIC will have even fewer specifications than BASIC. Having a small set of primitive types and limited data structures, forces the 
user to gain a deep foundational understanding of these concepts. This way, if they choose to continue programming, they have a solid 
understanding of the foundations and it makes it easier to continue coding using more expansive languages like Python, Java, C, and F#.
Easier to reason about because of F#'s strong typing. 

\subsection*{Examples}

10 PRINT "Hello, World!"


----------

10 INPUT "Enter a number: ", A
20 INPUT "Enter another number: ", B
30 SUM = A + B
40 PRINT "The sum of "; A; " and "; B; " is "; SUM


—————

10 INPUT "Enter a number for counting: ", N
20 FOR I = 1 TO N
30   IF I % 2 = 0 THEN PRINT I; " is even"
40 NEXT I


\subsection*{Language Concepts}

Error 

\subsection*{Formal Syntax}

\todo[inline]{Delete this TODO and replace with BNF.}

\subsection*{Semantics}

\todo[inline]{Delete this TODO and replace with as much text as is needed.}

% DO NOT DELETE ANYTHING BELOW THIS LINE
\end{document}
