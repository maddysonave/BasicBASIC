\documentclass[10pt]{article}

% Lines beginning with the percent sign are comments
% This file has been commented to help you understand more about LaTeX

% DO NOT EDIT THE LINES BETWEEN THE TWO LONG HORIZONTAL LINES

%---------------------------------------------------------------------------------------------------------

% Packages add extra functionality.
\usepackage{times,graphicx,epstopdf,fancyhdr,amsfonts,amsthm,amsmath,algorithm,algorithmic,xspace,hyperref}
\usepackage[left=1in,top=1in,right=1in,bottom=1in]{geometry}
\usepackage{sect sty}   %For centering section headings
\usepackage{enumerate}  %Allows more labeling options for enumerate environments 
\usepackage{epsfig}
\usepackage[space]{grffile}
\usepackage{booktabs}
\usepackage{forest}
\usepackage{enumitem}   
\usepackage{fancyvrb}
\usepackage{todonotes}
\usepackage{longtable}

% This will set LaTeX to look for figures in the same directory as the .tex file
\graphicspath{.} % The dot means current directory.

\pagestyle{fancy}

\lhead{Final Project}
\rhead{\today}
\lfoot{CSCI 334: Principles of Programming Languages}
\cfoot{\thepage}
\rfoot{Spring 2024}

% Some commands for changing header and footer format
\renewcommand{\headrulewidth}{0.4pt}
\renewcommand{\headwidth}{\textwidth}
\renewcommand{\footrulewidth}{0.4pt}

% These let you use common environments
\newtheorem{claim}{Claim}
\newtheorem{definition}{Definition}
\newtheorem{theorem}{Theorem}
\newtheorem{lemma}{Lemma}
\newtheorem{observation}{Observation}
\newtheorem{question}{Question}

\setlength{\parindent}{0cm}

%---------------------------------------------------------------------------------------------------------

% DON'T CHANGE ANYTHING ABOVE HERE

% Edit below as instructed

\title{Basic BASIC Language Specification} % Replace SnappyLanguageName with your project's name

\author{Maddy Wu \and Amir Estejab} % Replace these with real partner names.

\begin{document}
  
\maketitle

\subsection*{Introduction}

As student's that took intro to CS at Williams, we learned how to program with Python. There were, of course, many benefits to learning 
how to code this way. For example, Python syntax, at the level that we were learning it, was not difficult to learn. However, there were 
also many challenges -- especially for individuals that have never coded before or have only had experience coding with languages like 
Scratch. One of the main challenges that we came across as students in 134, was trying to gain a solid foundational understanding of the 
specifications in a language. Python is obviously a large language with many specifications which is daunting to people that have never
programmed before. \\
\\
BASIC is a programming language that was originally designed to remedy this problem. Designed by John G. Kemeny and Thomas E. Kurtz at 
Dartmouth College in 1963 as a way to make it easy for non-STEM students to learn how to code. For our project, we are looking to create 
an even simpler version of the language that will hopefully further lower the barriers of entry to learning how to the program and really 
reinforce what we believe to be the foundations of programming. 

\subsection*{Design Principles}

Because Basic BASIC is meant to be accessible even to most non-technical of users, both our primitive types and our combining forms will 
try to resemble plain English as much as possible. Furthermore, Basic BASIC will have even fewer specifications than BASIC which will 
hopefully be conducive to ease of learning and a stronger understanding of "foundational" concepts. If the user chooses to continue 
programming. BASIC Basic will lend itself nicely to other more complex languages. 

\subsection*{Examples}
\begin{enumerate}
   
    \item
    test1.txt:

    \begin{verbatim}
    a = true
    b = false 
    IF b THEN PRINT "hello" ELSE PRINT "goodbye"
    \end{verbatim}

    Run ”dotnet run test1.txt” in the project file. The output should be the following:

    \begin{verbatim}
    "goodbye"
    \end{verbatim}

    \item
    test-2.txt:
    
    \begin{verbatim}    
    a = 23
    b = 2 ^ 3
    c = 2 + 3
    d = 2 - 3
    e = 2 * 3
    
    PRINT a
    PRINT b
    PRINT c
    PRINT d
    PRINT e 
    \end{verbatim}

    Run ”dotnet run test2.txt” in the project file. The output should be the following: \\

    \begin{verbatim}
    23
    8
    5
    -1
    6
    \end{verbatim}

    \item
    test-3.txt:

    \begin{verbatim}    
        l = 5
        w = 3
        a = l * w
        p = 2 * (l + w)
        
        PRINT "Rectangle Properties:"
        PRINT l
        PRINT w
        PRINT a 
        PRINT p
    \end{verbatim}

    Run ”dotnet run test3.txt” in the project file. The output should be the following: \\

    \begin{verbatim}
    5
    3
    15
    16
    \end{verbatim}
\end{enumerate}

\subsection*{Language Concepts}

In order to use BASIC Basic the user should have an understanding of basic $($haha$)$ math operations. They also need to have some an 
understanding of strings and print statements. Strings are a primitive data type in this language meaning they cannot be broken down 
into constiuent parts. The print statment as well as the arithmetic operators and comparison operations, on the other hand, are 
combining forms. \\
\\
In Basic BASIC we also have conditionals and GO TO statements. Although GO TOs are technically considered bad practice when coding, 
we chose to implement them because it is the most simplistic way of introducing control flow in programs. It allows the user to think 
more like a computer, by stepping through a program the same way that a computer would, without any previous or advanced programming 
knowledge.\\
\\


\subsection*{Formal Syntax}

\begin{verbatim}
<Expr>        ::= <Statement> 
                | <Statement> <Expr>
<Statement>   ::= <Assignment> | <PRINT> | <Conditional> | <Arithmetic> | <Comparison>  
<Assignment>  ::= <Var> '=' <Primitive>
<Var>         ::= [a-zA-Z]+
<Primitive>   ::= <Bboolean> | <Bstring> | <Num>
<Bboolean>    ::= True | False 
<Bstring>     ::= " " 
<Num>         ::= n\in\Z
<PRINT>       ::= PRINT <Expr>
<Arithmetic>  ::= (<Expr> + <Expr>) | (<Expr> - <Expr>) | 
                  (<Expr> * <Expr>) | (<Expr> / <Expr>) | 
                  (<Expr> ^ <Expr>)
<Conditional> ::= IF <Statement> THEN <Statement> ELSE <Statement>
\end{verbatim}

\subsection*{Semantics}

    \begin{tabular}{ |p{3cm}|p{3cm}|p{3cm}|p{3cm}|p{3cm}|  }
    \hline
    \multicolumn{5}{|c|}{Semantics} \\
    \hline
    Syntax& Abstract Syntax& Type& Prec./Assoc.& Meaning \\
    \hline
    "s" &Bstring of string &string &n/a & A sequence of characters enclosed in double quotes ("). It is a primitive \\
    \hline
    true/false &Bbool of bool &string &n/a & Truth values indicating whether the result of a comparitive operation is true or false. It is a primitive \\
    \hline
    n &Num of int &int &n/a &N is any positive or negative integer. It is a primitive. \\
    \hline
    x &Var of string &int &n/a &N is any positive or negative integer. It is a primitive. \\
    \hline
    = &Assignment of string * Expr &char &n/a & Assigns the value of the right expression to the value of the left expression \\
    \hline
    + &Plus of Expr * Expr &char &left-associative & Adds the value of the left expression to the value of the right expression.\\
    \hline
    - &Minus of Expr * Expr &char &left-associative & Subtracts the value of the right expression from the value of the left expression.\\
    \hline
    * &Times of Expr * Expr &char &left-associative & Multiplies the value of the left expression to the value of the right expression.\\
    \hline
    / &Divide of Expr * Expr &char &left-associative & Divides the value of the left expression by the value of the right expression.\\
    \hline
    \^\ &Exp of Expr * Expr &char &right-associative & Raises the value of the left expression to the value of the right expression.\\
    \hline
    $()$ &Parens of Expr &char &n/a & parentheses\\
    \hline
    IF THEN &IfThen of Expr * Expr &string &n/a & Checks for a condition and then either returns a value or moves on to the next block of code.\\
    \hline
    IF... THEN... ELSE&IfThenElse of Expr * Expr * Expr &string &n/a & Extension of If...Then. Checks for a condition and then either returns a value or moves on to the next block of code.\\
    \hline
    \end{tabular}
\\

\begin{enumerate}
    \item
    What are the primitive kinds of values in your system? For example, a primitive might be a number, 
    a string, a shape, or a sound. Every primitive should be an idea that a user can explicitly state 
    in a program written in your language. 
    \begin{enumerate}
        \item
        The primitives we have are strings, numbers, and booleans
    \end{enumerate} 
    \item 
    What are the combining forms in your language? In other words, how are values combined in
    a program? For example, your system might combine primitive “numbers” using an operation
    like “plus.” Or perhaps a user can arrange primitive “notes” within a “sequence.”
    \begin{enumerate}
        \item 
        The combining forms we have are your typical arithmetic operators like addition, subtratction,
        multiplication, division, and exponentiation. 
    \end{enumerate} 
    \item 
    How is your program evaluated? In particular
    \begin{enumerate}
        \item 
        Do programs in your language read any input?
        \begin{enumerate}
            \item
            As of right now our program does not take in user input, but
            at some point it is something that we would like to implement as we believe that it will make 
            programming with Basic BASIC more dynamic. 
        \end{enumerate}
        \item
        What is the effect (output) of evaluating a program? Does the language produce a file or
        print something to the screen? Use one of your example programs to illustrate what you
        expect as output.
        \begin{enumerate}
            \item
            Depending on the program that the user writes, Basic BASIC should be able to interpret any
            program written in the language and output what the user is hoping to output. The most 
            complex example we have provided is a game called STM which should generate a playable game. 
        \end{enumerate}
    \end{enumerate}
\end{enumerate}

\subsection*{Remaining Work and Limitations}
Unfortunately, in its current state Basic BASIC is rather limited in what it is able to accomplish. We were hoping to implement 
GO TO statements that would establish control logic into our program, but due to some issues that we experienced while trying to
implement conditionals we did not get a chance to move to GO TOs. Although it was not shown in our presentation, we were able to 
parse line numbers successfully, but as line numbers only come in to play with GO TOs we also had to scrap the idea to focus on 
conditionals. \\
\\
The nearest reach goal for Basic BASIC, is to create a full BASIC interpreter with FOR... NEXT, GO TO, line numbers, END,
reading input and all that was laid out in the original language. Once a full BASIC interpreter is implemented, it would be nice 
to tinker with it, and make it more adaptable to modern day programmers. Since it's creation in 1964, BASIC has been pretty much 
rendered as obsolete. It would be interesting if BASIC were able to be revived as perhaps an introduction to C programming, as 
we have discovered, BASIC functions in a similar manner to C.


% DO NOT DELETE ANYTHING BELOW THIS LINE
\end{document}